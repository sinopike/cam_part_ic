\documentclass[14pt]{article}
\usepackage[utf8]{inputenc}
\usepackage[margin=1.2in]{geometry}
\geometry{a4paper} 

\usepackage{booktabs} 
\usepackage{array} 
\usepackage{paralist} 
\usepackage{verbatim} 
\usepackage{subfig} 
\usepackage{graphicx} 
\usepackage{amsmath}
\usepackage{amssymb}
\usepackage{bm}
\usepackage{bbm}
\usepackage{indentfirst}

\usepackage{fancyhdr}
\pagestyle{fancy}
\renewcommand{\headrulewidth}{0pt}
\lhead{}\chead{}\rhead{}
\lfoot{}\cfoot{\thepage}\rfoot{}

\usepackage{sectsty}
\allsectionsfont{\sffamily\mdseries\upshape}

\usepackage[nottoc,notlof,notlot]{tocbibind} % Put the bibliography in the ToC
\usepackage[titles,subfigure]{tocloft} % Alter the style of the Table of Contents
\renewcommand{\cftsecfont}{\rmfamily\mdseries\upshape}
\renewcommand{\cftsecpagefont}{\rmfamily\mdseries\upshape} % No bold!

\newcommand{\sech}{\text{sech}}
\newcommand{\Bxi}{\boldsymbol{\xi}}

\newcommand{\const}{\text{ const }}
\newcommand{\as}{\text{ as }}
\newcommand{\e}{\text{e}}
\newcommand{\Grad}{\boldsymbol{\nabla}}
\newcommand{\Vort}{\boldsymbol{\omega}}
\newcommand{\Angu}{\boldsymbol{\Omega}}
\newcommand{\pard}[2]{\frac{\partial #1}{\partial #2}}
\newcommand{\pardd}[2]{\frac{\partial^2 #1}{\partial #2^2}}
\newcommand{\pardda}[3]{\frac{\partial^2 #1}{\partial #2 \partial #3}}
\newcommand{\newd}[2]{\frac{\text{d} #1}{\text{d} #2}}
\newcommand{\newdd}[2]{\frac{\text{d}^2 #1}{\text{d} #2^2}}
\newcommand{\lagd}[2]{\frac{\text{D} #1}{\text{D} #2}}

\newcommand{\so}{\text{so }\:\:\:}
\newcommand{\So}{\text{So }\:\:\:}

\newcommand{\hence}{\text{hence }\:\:\:}
\newcommand{\Hence}{\text{Hence }\:\:\:}

\newcommand{\since}{\text{since }\:\:\:}
\newcommand{\Since}{\text{Since }\:\:\:}

\newcommand{\wehave}{\text{we have }\:\:\:}
\newcommand{\Wehave}{\text{We have }\:\:\:}

\newcommand{\wealsohave}{\text{we also have }\:\:\:}
\newcommand{\Wealsohave}{\text{We also have }\:\:\:}
\newcommand{\AU}{\text{AU}}
\newcommand{\eqh}[1]{\text{#1 }\:\:\:}
\newcommand{\Prob}[1]{\mathbb{P}(#1) }
\newcommand{\prob}{\mathbb{P}}
\newcommand{\var}{\text{Var}}
\newcommand{\cov}{\text{Cov}}
\newcommand{\corr}{\text{Corr}}



\title{Stochastic Processes\\{\small Answer Sheet 1, Due 24/10/2019}}
\author{Yuchen Pei}
\date{}

\begin{document}
\maketitle




\section*{1}

Given

$$
(\Omega,\mathcal{A},\mu)\text{ is a measure space}
$$

Consider $\{N_i\}$ with $i\in\mathbb{N}$ such that $\mu(N_i)=0\forall i\in\mathbb{N}$

$$
\mu(\bigcup_{i=1}^{\infty}N_i)=\sum_{i=1}^{\infty}\mu(N_i)-\mu(\text{intersections})\leq\sum_{i=1}^{\infty}\mu(N_i)=0
$$

$$
\mu(A)\geq 0\forall A\in\mathcal{A},\mu(\bigcup_{i=1}^{\infty}N_i)\leq 0\implies\mu(\bigcup_{i=1}^{\infty}N_i)=0\implies \bigcup_{i=1}^{\infty}N_i\text{ is a null set.}
$$

\bigskip
\section*{2}

Given

$$
\mathcal{S}=\{
(a,b]:-\infty\leq a \leq b <\infty
\}
$$

\subsection*{(a)}

$(a,a]=\emptyset\implies\emptyset\in\mathcal{S}$

$
\forall(a,b],(c,d]\in\mathcal{S}\:(\text{WLOG, let }a<c):(a,b]\cap(c,d]=\left\{
\begin{array}{ll}
(c,b]\in\mathcal{S} &\text{ if }c<b<d
\\
(c,d]\in\mathcal{S} &\text{ if }c<d\leq b
\\
\emptyset\in\mathcal{S} &\text{ otherwise}
\end{array}
\right.
$

$
\forall(a,b],(c,d]\in\mathcal{S}:(a,b]\setminus(c,d]=\left\{
\begin{array}{ll}
(d,b]\in\mathcal{S} &\text{ if }c<a<d<b
\\
(a,c]\in\mathcal{S} &\text{ if }a<c<b<d
\\
(a,c]\cup(d,b] \text{ with } (a,c],(d,b]\in\mathcal{S} &\text{ if }a\leq c<d\leq b
\\
(a,b]\in\mathcal{S} & \text{ otherwise }
\end{array}
\right.
$

Therefore $\mathcal{S}$ is a semi-ring.

\subsection*{(b)}

Given

$$
\mu((a,b])=b-a
$$

Consider arbitrary disjoint intervals $(a,b]$, $(c,d]$, WLOG , let $c\geq b$, such that $(a,b]\cup(c,d]\in\mathcal{S}$

If $a=b$, then $\mu((a,b]\cup(c,d])=\mu((c,d])+0=\mu((a,b])+\mu((c,d])$.

Similarly if $c=d$, then $\mu((a,b]\cup(c,d])=\mu((a,b])+0=\mu((a,b])+\mu((c,d])$.

Otherwise we must have $b=c$, so that $(a,b]\cup(c,d]=(a,d]$

$\implies
\mu((a,b]\cup(c,d])=d-a=d-c+c-a=\mu((a,b])+\mu((c,d])$.

\bigskip

Consider disjoint $\{(a_i,b_i]\}_{1\leq i\leq n}$ such that $\bigcup_{i=1}^{n}(a_i,b_i]\in\mathcal{S}$.

WLOG, let $b_i\leq a_{i+1}\forall i$ then we must also have $a_{i+1}=b_{i}\forall i \in \mathbb{N}$ so that $\bigcup_{i=1}^{n}(a_i,b_i]=(a_1,b_n]\in\mathcal{S}$

Now suppose that

$$
\mu\left(
\bigcup_{i=1}^{n}
(a_i,b_i]
\right)
=
\sum_{i=1}^n
\mu((a_i,b_i])
$$

 


Then by what we proved before:

$$
\mu\left(
\bigcup_{i=1}^{n}
(a_i,b_i]
\cup
(a_i+1,b_i+1]
\right)
=
\sum_{i=1}^{n}
\mu((a_i,b_i])+\mu((a_i+1,b_i+1])
$$

$$
\implies
\mu\left(
\bigcup_{i=1}^{n+1}
(a_i,b_i]
\right)
=
\sum_{i=1}^{n+1}
\mu((a_i,b_i])
$$

Where $a_{n+1}=b_n$ so that $\bigcup_{i=1}^{n+1}(a_i,b_i]=(a_1,b_n]\in\mathcal{S}$ 

Therefore by induction, $\mu$ is finitely additive.

\subsection*{(c)}

Consider disjoint $\{(a_i,b_i]\}_{1\leq i\leq n}$ such that $\bigcup_{i=1}^{\infty}(a_i,b_i]\in\mathcal{S}$.

WLOG, let $b_i\leq a_{i+1}\forall i$ then we must also have $a_{i+1}=b_{i}\forall i$ so that $\bigcup_{i=1}^{n}(a_i,b_i]=(a_1,b_\infty]\in\mathcal{S}$. Here $b_\infty$ is the limit of $\{b_i\}$ since we require it to be finite and $\{b_i\}$ to be increasing then $\{b_i\}$ must converge.

$$
\mu\left(
\bigcup_{i=1}^{\infty}
(a_i,b_i]
\right)
=
b_\infty-a_1
=
a_\infty-a_1
\text{ since } a_{i+1}=b_{i}\forall i
$$

$$
\sum_{i=1}^{\infty} \mu((a_i,a_{i+1}])=
\sum_{i=1}^{\infty} a_{i+1}-a_i=
\lim_{n\rightarrow\infty}(\sum_{i=2}^{n+1} a_{i} -\sum_{i=1}^{n}a_i)=
\lim_{i\rightarrow\infty} a_{i+1}-a_1=
a_\infty-a_1
$$

$$
\implies
\mu\left(
\bigcup_{i=1}^{\infty}
(a_i,b_i]
\right)
=\sum_{i=1}^{\infty} \mu((a_i,a_{i+1}])
$$

Therefore $\mu$ is countably additive.

Since we also have $\mu(\emptyset)=0$, $\mu$ is a pre-measure.


\bigskip
\section*{3}

\subsection*{(a)}

Suppose that $\mathcal{G}$ is a $\sigma$-field, then

$\Omega\in\mathcal{G}$

$\bigcap_{i=1}^n g_i=(\bigcup_{i=1}^n(g_i^c))^c\in\mathcal{G}\:\:\:\forall n\in\mathbb{N}, \{g_i\}\in\mathcal{G}$

$A\setminus B=A\cap B^c\in\mathcal{G}\:\:\: \forall A,B\in\mathcal{G}:A\supseteq B $

$\bigcup_{i=1}^{\infty}A_i\in\mathcal{G}\:\:\:\forall A_n\in\mathcal{G}:A_i\subseteq A_{i+1}\forall i$

Then $\mathcal{G}$ is both a d-system and a $\pi$-system.

\bigskip

Suppose $\mathcal{G}$ is both a d-system and a $\pi$-system, then

$\Omega\in\mathcal{G},\emptyset=A\setminus A\in\mathcal{G}\text{ for some } A\in\mathcal{G}$

$A^c=\Omega\setminus A\in\mathcal{G}\forall A \in\mathcal{G}$

$\bigcup_{i=1}^{\infty} A_i=\bigcup_{i=1}^{\infty}\bigcup_{j=1}^{i} A_j=\bigcup_{i=1}^{\infty}B_i\in\mathcal{G}\:\:\:\forall \{A_i\}\in\mathcal{G} $ since $B_i\subseteq B_{i+1}\forall i$

Then $\mathcal{G}$ is a $\sigma$-field.

\subsection*{(b)}

Given

$$
\mathcal{G}\text{ is a }\pi-\text{system.}
$$

Suppose $\mathcal{S}_1$, $\mathcal{S}_2$ are sigma field on set $\Omega$.

$$\text{Then }\begin{array}{lr}
\left\{
\begin{array}{l}
\emptyset,\Omega\in\mathcal{S}_j
\\
 A^c\in\mathcal{S}\forall A\in\mathcal{S}_j
\\
\bigcup_{i=1}^{\infty} A_i\in\mathcal{S}_j\forall \{A_i\}\in\mathcal{S}_j
\end{array}\right.
&\forall j\in\{1,2\}
\end{array}
$$

$$
\implies \left\{
\begin{array}{l}
\emptyset,\Omega\in\mathcal{S}
\\
 A^c\in\mathcal{S}\forall A\in\mathcal{S}
\\
\bigcup_{i=1}^{\infty} A_i\in\mathcal{S}\forall \{A_i\}\in\mathcal{S}_j
\end{array}\right.
\text{ where }\mathcal{S}=\mathcal{S}_1\cap\mathcal{S}_2
$$ 

$$
\implies
\mathcal{S}_1\cap \mathcal{S}_2
\text{ is a }\sigma\text{-field }\forall \sigma\text{-fields }\mathcal{S}_1,\mathcal{S}_2
$$

$d(\mathcal{G})$ is both a d-system and a $\pi$-system $\implies$ $d(\mathcal{G})$ is a $\sigma$-field.

Then $\sigma(\mathcal{G})\cap d(\mathcal{G})$ is also a $\sigma$-field, and hence a d-system as well.

$\sigma(\mathcal{G})\cap d(\mathcal{G})\subseteq\sigma(\mathcal{G}), \sigma(\mathcal{G})\cap d(\mathcal{G})\supseteq\sigma(\mathcal{G}) \implies \sigma(\mathcal{G})\cap d(\mathcal{G})=\sigma(\mathcal{G})$ since $\sigma(\mathcal{G})$ is the smallest $\sigma$-field containing $\mathcal{G}$.

$$
\implies 
\sigma(\mathcal{G})= d(\mathcal{G})
$$

\bigskip
\section*{4}

For $\{a_n\}_{n=1}^{\infty}$:

$$
\limsup_{n\rightarrow\infty} a_n=
\inf_{n\geq 1} \sup_{k\geq n}a_k\in[-\infty,\infty]
$$

$$
\liminf_{n\rightarrow\infty}a_n=
\sup_{n\geq 1}\inf_{k\geq n}a_k\in[-\infty,\infty]
$$


\subsection*{(a)}

Given $\{a_n\}_{n=1}^{\infty}$, with $a_i\in\mathcal{P}(\Omega)\forall i$, define

$$
\limsup A_n = \bigcap_{n=1}^{\infty}\bigcup_{k=n}^{\infty} A_k
,\:\:
\liminf A_n = \bigcup_{n=1}^{\infty}\bigcap_{k=n}^{\infty} A_k
$$

\begin{align*}
x\in\bigcap_{n=1}^{\infty}\bigcup_{k=n}^{\infty} A_k
\iff & \forall n \exists k\geq n:
x\in A_k 
\\
\iff & x \text{ belongs to infinitely many }A_k
\end{align*}

Therefore $\limsup A_n=\{
x\in\Omega: x\text{ belongs to infinitely many }A_k
\}$

\begin{align*}
x\in\bigcup_{n=1}^{\infty}\bigcap_{k=n}^{\infty} A_k
\iff &
\exists n:\forall k\geq n: x\in A_k
\\ \iff & x \text{ belongs to all but finitely many }A_n
\end{align*}

Therefore $\liminf A_n=\{
x\in\Omega: x\text{ belongs to all but finitely many }A_k
\}$

\subsection*{(b)}

\begin{align*}
\mathbbm{1}_{\limsup A_n}(x)=1
\iff & x\in\limsup A_n
\\
\iff & \forall n \exists k\geq n: x\in A_k 
\end{align*}

\begin{align*}
\limsup \mathbbm{1}_{A_n}(x)=1
\iff & \inf_{n\geq 1} \sup_{k\geq n} \mathbbm{1}_{A_k}(x)=1
\\
\iff & \forall n \exists k\geq n: x\in A_k 
\\
\iff & \limsup \mathbbm{1}_{A_n}(x)=1
\end{align*}

Therefore $\mathbbm{1}_{\limsup A_n}=\limsup \mathbbm{1}_{A_n}$, since the indicator function is either 1 or 0.

\begin{align*}
\mathbbm{1}_{\liminf A_n}(x)=1
\iff & x\in\liminf A_n
\\
\iff & \exists n :\forall k\geq n: x\in A_k 
\end{align*}

\begin{align*}
\liminf \mathbbm{1}_{A_n}(x)=1
\iff & \sup_{n\geq 1} \inf_{k\geq n} \mathbbm{1}_{A_k}(x)=1
\\
\iff & \exists n :\forall k\geq n: x\in A_k 
\\
\iff & \liminf \mathbbm{1}_{A_n}(x)=1
\end{align*}

Therefore $\mathbbm{1}_{\liminf A_n}=\liminf \mathbbm{1}_{A_n}$, since the indicator function is either 1 or 0.

\end{document}
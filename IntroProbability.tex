\documentclass[14pt]{article}
\usepackage[utf8]{inputenc}
\usepackage[margin=1.2in]{geometry}
\geometry{a4paper} 

\usepackage{booktabs} 
\usepackage{array} 
\usepackage{paralist} 
\usepackage{verbatim} 
\usepackage{subfig} 
\usepackage{graphicx} 
\usepackage{amsmath}
\usepackage{amssymb}
\usepackage{bm}
\usepackage{bbm}
\usepackage{indentfirst}

\usepackage{fancyhdr}
\pagestyle{fancy}
\renewcommand{\headrulewidth}{0pt}
\lhead{}\chead{}\rhead{}
\lfoot{}\cfoot{\thepage}\rfoot{}

\usepackage{sectsty}
\allsectionsfont{\sffamily\mdseries\upshape}

\usepackage[nottoc,notlof,notlot]{tocbibind} % Put the bibliography in the ToC
\usepackage[titles,subfigure]{tocloft} % Alter the style of the Table of Contents
\renewcommand{\cftsecfont}{\rmfamily\mdseries\upshape}
\renewcommand{\cftsecpagefont}{\rmfamily\mdseries\upshape} % No bold!

\newcommand{\sech}{\text{sech}}
\newcommand{\Bxi}{\boldsymbol{\xi}}

\newcommand{\const}{\text{ const }}
\newcommand{\as}{\text{ as }}
\newcommand{\e}{\text{e}}
\newcommand{\Grad}{\boldsymbol{\nabla}}
\newcommand{\Vort}{\boldsymbol{\omega}}
\newcommand{\Angu}{\boldsymbol{\Omega}}
\newcommand{\pard}[2]{\frac{\partial #1}{\partial #2}}
\newcommand{\pardd}[2]{\frac{\partial^2 #1}{\partial #2^2}}
\newcommand{\pardda}[3]{\frac{\partial^2 #1}{\partial #2 \partial #3}}
\newcommand{\newd}[2]{\frac{\text{d} #1}{\text{d} #2}}
\newcommand{\newdd}[2]{\frac{\text{d}^2 #1}{\text{d} #2^2}}
\newcommand{\lagd}[2]{\frac{\text{D} #1}{\text{D} #2}}

\newcommand{\so}{\text{so }\:\:\:}
\newcommand{\So}{\text{So }\:\:\:}

\newcommand{\hence}{\text{hence }\:\:\:}
\newcommand{\Hence}{\text{Hence }\:\:\:}

\newcommand{\since}{\text{since }\:\:\:}
\newcommand{\Since}{\text{Since }\:\:\:}

\newcommand{\wehave}{\text{we have }\:\:\:}
\newcommand{\Wehave}{\text{We have }\:\:\:}

\newcommand{\wealsohave}{\text{we also have }\:\:\:}
\newcommand{\Wealsohave}{\text{We also have }\:\:\:}
\newcommand{\AU}{\text{AU}}
\newcommand{\eqh}[1]{\text{#1 }\:\:\:}
\newcommand{\Prob}[1]{\mathbb{P}(#1) }
\newcommand{\prob}{\mathbb{P}}
\newcommand{\var}{\text{Var}}
\newcommand{\cov}{\text{Cov}}
\newcommand{\corr}{\text{Corr}}



\title{Introduction to Probabilities\\{\small Lectured by I forgot his name}}
\author{Note taken by Yuchen Pei}
\date{}
\begin{document}
\maketitle

\section{Intuition}

\subsection{Event space}

$\Omega_D=\{ 1,2,3,4,5,6 \}$

$\Omega_{[0,1]^2}=\{ 0\leq x\leq 1, (x,y):0\leq x \leq 1, 0\leq y \leq 1 \}$

Events $A\subseteq \Omega$

$\Omega_D$: $A=\{ 1,2,3 \}$

$\Omega_{[0,1]^2}$: $A=\{0\leq x\leq 1/2, 1/2\leq y\leq 1\}$

\subsection{Probability measure}

$\mathbb{P},P$

$A \subseteq \Omega \rightarrow \Prob{A}$ probability that event A occurs

\begin{align*}
0\leq & \Prob{A} \leq 1
\\
&\Prob{\emptyset}=0
\\
&\Prob{\Omega}=1
\end{align*}

If $\mathbb{P}(A)=0$ we say A does not occur almost surely

If $\mathbb{P}(A)=1$ we say A does occur almost surely



\subsection{Examples}

D for dice

\begin{align*}
\Prob{D=3}&=1/6
\\
\Prob{D=2}&=1/6
\\
\Prob{D=2\text{ or }D=3}&=2/6
\\
\Prob{D\in\{2,3\}}&=2/6
\end{align*}

$A=\{D=2\}$, $B=\{D=3\}$

$
\Prob{A\cup B}=\Prob{A}+\Prob{B}
$
 for $A\cap B=\emptyset$

Counter example as $C\cap D\neq\emptyset$:

\begin{align*}
C=\{D=2,3\}
\\
D=\{D=3\}
\end{align*}

for
$
\{A_i\}
$
Pairwise disjoint events

$$
\Prob{ \bigcup_{i=-1}^{N} A_i }=\sum_{i=1}^{N}\Prob{A_i}
$$

for
$
(A_i)_{i\in I}
$
, I countable, pairwise disjoint events

$$
\Prob{ \bigcup_{i=-1} A_i }=\sum_{i=1}\Prob{A_i}
$$

$I=\mathbb{N}$

\subsection{More Examples}

$$
\Omega_D: \: \Prob{A}=\frac{ XX \text{E(events in A)}}{6}
$$

$$
\Omega_{[0,1]^2}: \: \Prob{A}=\frac{\text{Area}(A)}{\text{Area(Box)}=1}
$$

$$
\Omega_{[0,1]}: \: \Prob{A}\frac{\text{length}(A)}{1}
$$

$x=1/2$, $\Prob{x}=0$

counter example for uncountable $[0,1]$

\begin{align*}
1=\Prob{\Omega_{[0,1]}}
\\
=\Prob{\bigcup_{x\in[0,1]}\{x\}}
\\
=\sum_{{x\in[0,1]}}\Prob{x}
\\
=\sum_{{x\in[0,1]}}0
\\
=0
\end{align*}

$$
[0,1]=\{ x\in\mathbb{R}:0\leq x\leq 1 \}=A=\bigcup_{{x\in[0,1]}}\{x\}=B
$$

$$
\begin{matrix}
z\in A\implies z\in B
\\
z\in B\implies z\in A
\end{matrix}
\implies
A+B
$$

\subsection{Definitions}

We say $\prob$ is a probability measure on $\Omega$ if and only if

\begin{align*}
\Prob{\emptyset}&=0
\\
\Prob{\Omega}&=1
\\
\Prob{A}&\geq 0 \text{ for events }A\subseteq \Omega
\end{align*}

$A_i, i\in I$, countably many pairwise disjoint events

$$
\Prob{\bigcup_{i\in I}A_i}=\sum_{i\in I}\Prob{A_i}
$$

We call $(\Omega,\prob)$ a probability space

\subsection{Lemma}

$$
B\subseteq A\implies \Prob{B}\leq\Prob{A}
$$

\subsubsection{Proof}

\begin{align*}
\Prob{A}
&=\Prob{A\cap\Omega}
\\
&=\Prob{A\cap(B\cup \bar{B})}
\\
&=\Prob{(A\cap B)\cup (A\cap \bar{B})    }
\\
&=\Prob{ B\cup (A\cap\bar{B})   }
\\
&=\Prob{B}+\Prob{A\cap\bar{B}}
\\
&\geq \Prob{B}
\end{align*}





\section{Independence}

$
D_1, D_2 \rightarrow\text{two dice}
$

$$
\Omega_{D^2}=\{ (1,1), (1,2)\dots (1,6),(2,1)\dots(2,6)\dots(6,6)   \}
$$

$$
\Prob{A}=\frac{XX\text{E(events)}}{36}
$$

$
\Prob{D_1=2}=1/6,\: \Prob{D_2=3}=1/6
$

$
\Prob{D_1=2,D_2=3}=\Prob((2,3))=1/36
$


$A=\{D_1=2\}, B=\{D_2=3\}$

$$
\Prob{A}\cdot\Prob{B}=\Prob{A\cap B}
$$

(only for independent $A,B$)

$
\tilde{D}_1
\tilde{D}_2
:
\Omega_D=\{(1,1),(2,2),(3,3)\dots(6,6)\}
:
\Prob{(2,2)}=\frac{1}{6}
$

$
\Prob{(2,3)}=0
$

$
\Prob{\tilde{D}_1=\tilde{D}_2=1}=\Prob{(1,1)}=\frac{1}{6}
$

$
\Prob{\tilde{D}_1=1}=\frac{1}{6}
$

$
\Prob{\tilde{D}_2=1}=\frac{1}{6}
$

$
1/6 \neq 1/6-1/6
$

$$
A=\{  \tilde{D}_1=1\} \text{ and } B=\{ \tilde{D}_2=1 \} \text{ are not independent.}
$$

\subsection{Definitions}

We say the sets $A_1\dots A_N$ are independent if and only if for every subfamily $A_{i_1}\dots A_{i_k}$, $2<k<n$, $1\leq i_{1}<i_2\dots i_k\leq n$.

$$
\Prob{ \bigcap_{j=1}^{k} A_{i_j} }=\prod_{j=1}^{k} \Prob{A_{i_j}}
$$

For countably many events $(A_i)_{i\in I}$ we say they are independent if and only if every finite subfamily is independent.

$A\perp B\iff $ A is independent of B

$A\perp A \implies \Prob{A}\in\{0,1\}$

$$\text{as }
\Prob{A\cap A}=\Prob{A}\cdot\Prob{A}\implies x=x^2\implies x \in \{0,1\}
$$






\section{Conditional Probability}

\subsection{Definitions}

Given two events A and B such that $\Prob{B}>0$, then we say the conditional probability of $A$ with respect to $B$ is given by

$$
\Prob{A|B}=\frac{\Prob{A\cap B}}{\Prob{B}}
$$

$A\cap B=\emptyset\implies \Prob{A|B}=0$

$\Prob{A|A}=1$

$\Prob{A|\Omega}=\Prob{A}$

\subsection{Baye's rule}

Given a partition $(A_i,i\in\{1,2\dots N\})$ such that $\forall i \Prob{A_i}>c$ and an event $B$ such that $\Prob{B}>0$, then

$$
\Prob{A_i|B}=\frac{\Prob{A_i\cap B}}{\Prob{B}}=\frac{\Prob{A_i}\Prob{B|A_i}}{\sum_{j=-1}^N\Prob{A_j}\Prob{B|A_j}}
$$

\section{Random variables}

$$
\Omega_1=\{\text{H,T}\}
\begin{matrix}
+1 if \text{H}
\\
-1 if \text{T}
\end{matrix}
$$

$$
\Omega_D=\{ 1,2,3,4,5,6 \}
$$

\subsection{Definitions}

Given a probability space $(\Omega,\prob)$ we call a function $X \Omega\mapsto \mathbb{R}$ as random variable

$$
\Omega_1 \bigg|
\begin{matrix}
X(\text{H})=1
\\
X(\text{T})=-1
\end{matrix}
$$

$
\mathbbm{1}_A\Omega\rightarrow \mathbb{R}
$

$$
\mathbbm{1}_A(\omega)=\bigg\{
\begin{matrix}
1 \text{ if }\omega \in A
\\
0 \text{ if }\omega \not\in A
\end{matrix}
$$

\subsection{Distribution}

$\{x \in A\}\subseteq \Omega$

$\mu(A\subseteq\mathbb{R})=\Prob{X\in A}=\Prob{\{ \omega\in\Omega,  X(\omega)\in A \}}$

\subsubsection{Discrete}

Probability mass function (PMF): $P_X$

$
P_X(x)=\mu({x})=\Prob{X\in x}
$

Simple analysis shows

$$
P_X(x)=0 \text{ for all }x \in \mathbb{R} \text{ except countably many places.}
$$

example omitted

\subsubsection{Continuous}

Probability density function (PDF): $f_X$

$$
\Prob{X=x}=f_X(x)\text{d}x
$$

$$
\Prob{X\in A}=\int_A f_X(x)\text{d}x
$$

example omitted

Cumulative distribution function (CDF): $F_X$

$F_X(x)=\mu( (\infty,x] )$

Continuous case:

$$
F_X(x)=\int_{-\infty}^{x}f_X(\xi)\text{d}\xi
$$

Discrete Case:

$$
F_X(x)=\sum_{a_i\leq X}P_X{a_i}
$$

\bigskip

\subsection{Equality}

$$
\bar{X}:\Omega_\Lambda=\{H,T\}\mapsto \mathbb{R}
$$
$$
\bar{X}(H)=1, \bar{X}{T}=-1
$$

$$
\tilde{X}:\Omega_\Lambda=\{1,2,3,4,5,6\}\mapsto \mathbb{R}
$$
$$
\tilde{X}(1)=\tilde{X}{2}=\tilde{X}{3}=1,
\tilde{X}(4)=\tilde{X}{5}=\tilde{X}{6}=-1
$$

\begin{align*}
\mu_{\bar{X}} (A \subseteq \mathbb{R})
&=\Prob{ \bar{X}\in A }
\\
&=\Prob{ \{ \omega\in\Omega_\lambda \}:\bar{X}(\omega)=A}
\end{align*}

\begin{align*}
\mu_{\tilde{X}} (A\subseteq\mathbb{R})
&=\Prob{ \tilde{X}\in A }
\\
&=\Prob{\{\omega\in\Omega_\lambda\}:\tilde{X}(\omega)=A}
\end{align*}

$$
\forall A \subset R, \mu_{\bar{X}}(A)=\mu_{\tilde{X}}(A)
$$

\bigskip

Take any subset $A\subseteq \mathbb{R}$:

$$
\{ \omega\in\Omega_{\Lambda}:\bar{X}(\omega)\in A \}=
\Bigg\{
\begin{matrix}
\emptyset \text{ if }-\Lambda\not\in A \text{ and }1 \neq A \implies\mu_{\bar{X}}(A)=0
\\
\text{H} \text{ if }-\Lambda\not\in A \text{ but } 1\in A \implies\mu_{\bar{X}}(A)=1/2
\\
\text{T} \text{H} \text{ if }-\Lambda\not\in A \text{ but }1 \in A \implies\mu_{\bar{X}}(A)=1/2
\\
\text{H,T} \text{ if }-\Lambda\in A \text{ and }1 \in A\rightarrow \mu_{\bar{X}}(A)=1
\end{matrix}
$$

\section{Multiple random variables}

$$
X:\Omega \mapsto \mathbb{R}
$$

$$
(X_i)_{i\in I}:\Omega \mapsto \mathbb{R}
$$

e.g.: $X=$ Height, $Y$=Weights

\subsection{Some distributions}

\begin{align*}
(X,Y)&\mapsto \mathbb{R}^2
\\
\mu_{X,Y}(A^{u^{IR}})\times B^{u^{IR}})&=\Prob{X\in A, Y\in B}
\\
&=\Prob{\{ \omega\in\Omega: X(\omega)\in A \text{ and }Y(\omega\in B) \}}
\end{align*}

\subsection{Properties}

Joint PMF: $P_{X,Y}(x,y)=\Prob{X=x,Y=y}$

Joint PDF: $f_{X,Y}(x,y)=\Prob{X=x,Y=y}\text{d}x\text{d}y$

\subsubsection{CDF}

$$
F_{X,Y}(x,y)=\sum_{\tilde{x}\leq X, \tilde{y}\leq y} P_{X,Y}(x,y)
$$

$$
F_{X,Y}(x,y)=\int_{\tilde{x}\leq X, \tilde{y}\leq y} f_{X,Y}(x,y) \text{d}\tilde{d},\text{d}\tilde{y}
$$

\subsubsection{Marginal}

$$
F_{X,Y}(x,y)\rightarrow F_X(x)
$$

$$
F_X{x}=\Prob{X\leq x}=\Prob(X\leq x Y\leq \infty)
$$

$$
F_X{x}=F_{X,Y}(x,\infty)
$$

\subsubsection{Identities}

$$
P_X{x}=\sum_y P_{X,Y}(x,y)
$$

$$
f_X(x)=\int_y f_{X,Y}(x,y)\text{d}y
$$


\subsubsection{Conditionals}

\begin{align*}
P_{X|Y}(x|y)&=\Prob{X\leq x|Y=y}
\\&=\frac{\Prob{X=x,Y=y}}{\Prob{Y=y}}
\\&=\frac{P_{X,Y}(x,y)}{P_Y(y)}\text{ Discrete case }
\\&=\frac{f_{X,Y}(x,y)}{f_Y(y)}\text{ Continuous case }
\end{align*}

\subsubsection{Law of total probabilities}

$$
f_X(x)=\int_\mathbb{R} f_{X|Y}(x|y) f_Y(y)\text{d}y
\text{ Discrete case }
$$

$$
P_X(x)=\sum_y P_{X|Y}(x|y) P_Y(y)
\text{ Continuous case }
$$

\subsubsection{Continuous Baye's Rule}

$$
f_{X,Y}(x,y)=\frac{f_{Y|X}(y|x)f_X(x)}{\int_\mathbb{R}f_{Y|X}(y|\bar{x})f_X(\bar{x})\text{d}\bar{x}}
$$

\section{Independence of random variable}

\subsection{Definitions}

Given random variable $X_1\dots X_N:\Omega\mapsto \mathbb{R}$, we say they are independent if and only if:

$$
\forall A_i\in \mathbb{R}:
\Prob{ X_1\in A_1, X_2\in A_2\dots, X_N\in A_N }=\prod_{i=1}^{N}\Prob{X_i\in A_i}
$$

For countably many $(X_i)_{i\in I}$ we say they are independent $\perp$ if and only if every subfamily is independent:

$$
\Prob{X_1\leq x_1,\dots,X_N\leq x_N}=\prob_{i=1}^{N}\Prob{X_1\leq x_i}
$$

for every $(X_1,\dots,X_N)\in\mathbb{R}^N$

\bigskip

Discrete case, joint PMF:

$$
P_{X_1,\dots,X_N}=\prod_{i=1}^{N}P_{X_i}(x_i)
$$

for every $(X_1,\dots,X_N)\in\mathbb{R}^N$

\bigskip

Continuous case, joint PDF:

$$
f_{X_1,\dots,X_N}=\prod_{i=1}^{N}f_{X_i}(x_i)
$$

for every $(X_1,\dots,X_N)\in\mathbb{R}^N$

The three equations above are equivalent.

\bigskip

Take $A_1,\dots,A_N\in\mathbb{R}$

\begin{align*}
\Prob{x_1\in A_1,\dots,X_N\in A_N}
=&
\int_{A_1}\dots\int_{A_N}f_{X_1,\dots,X_N}(x_1,\dots,x_N)\text{d}x_1\dots\text{d}x_N
\\=&
\int_{A_1}\dots\int_{A_N} f_{X_1}(x_1)\dots f_{X_N}(x_N)\text{d}x_1\dots\text{d}x_N
\\=&
\int_{A_1}f_{X_1}(x_1)\text{d}x_1\cdot\dots\cdot \int_{A_N}f_{X_N}(x_N)\text{d}x_N
\\=&
\prod_{i=1}^{N}\Prob{x_1\in A_i}
\end{align*}

\section{Expectation}

$$
\mathbb{E}[X]=
\bigg\{
\begin{matrix}
\sum_x x\cdot\Prob{X=x}\text{ Discrete case }
\\
\int_\mathbb{R} xf_X{x}\text{d}x\text{ Continuous case }
\end{matrix}
$$

examples omitted

\subsection{Definitions}

We say a random variable is integrable if and only if

$$
\mathbb{E}[|X|]<\infty
$$

$$
\mathbb{E}[g(x)]=
\bigg\{
\begin{matrix}
\sum_x g(x)\Prob{X=x}\text{ Discrete case }
\\
\int_{\mathbb{R}}g(x)f_X(x)\text{d}x\text{ Continuous case }
\end{matrix}
$$

\subsection{Properties}

$
X=0 \text{ constantly } \implies \mathbb{E}[X]=0
$

$
X,Y\text{ integrable and }X\leq Y\text{ constantly }\implies \mathbb{E}[X]\leq\mathbb{E}[Y]
$

$
\mathbb{E}[aX+bY]=a\mathbb{E}[X]+b\mathbb{E}[Y]
$

$
|\mathbb{E}[X]|\leq\mathbb{E}[|X|]
$

$
X\perp Y\implies\mathbb{E}[XY]=\mathbb{E}[X]\mathbb{E}[Y] \textbf{ Exercise}
$

\bigskip

\subsection{More Definitions}

$\mathbb{E}[X^r]$ is called the $r$th moment

$\mathbb{E}[|X|^r]$ is called the $r$th absolute moment

$\mathbb{E}[(X-\mathbb{E}[X])^r]$ is called the $r$th centred moment. When $r=2$, this is variance $\var[X]$.

$\mathbb{E}[|X-\mathbb{E}[X]|^r]$ is called the $r$th absolute centred moment

$
\var[X]=\mathbb{E}[(X-\mathbb{E}[X])^2]
$

$
\sigma[X]=\sqrt{\var[X]}=\sqrt{\mathbb{E}[(X-\mathbb{E}[X])^2]}
$


Examples omitted

\section{Covariance}

\subsection{Definitions}

Given two random variables $X,Y$ we define the covariance to be

$$
\cov(X,Y)=\mathbb{E}[ (X-\mathbb{E}[X])(Y-\mathbb{E}[Y]) ]
$$

$$
\cov(X,X)=\var[X]
$$

$$
\corr(X,Y)=\frac{\cov(X,Y)}{\sigma[X]\sigma[Y]}
\implies
-1\leq\corr(X,Y)\leq 1
$$

$$
X\perp Y\implies \cov(X,Y)=0\textbf{ Exercise}
$$

\subsection{Properties}

$$
\cov(X,aY+bZ)=a\cov(X,Y)+b\cov(X,Z)
$$

$$
\cov(X,Y)=\cov(Y,X)
$$

$$
\var(\sum_{i=1}^N X_i)=\sum_{i=1}^N\var(X_i)+\sum_{i\neq j}(\cov(X_i,X_j))
$$

\section{Conditional Expectation}

\subsection{Definitions}

$$
\mathbb{E}[X|B]=\sum_x x\Prob{X=x|B}
$$

Special case $B={Y=y}$

$$
\mathbb{E}[X|Y=y]=
\bigg\{
\begin{matrix}
\sum_x x\Prob{X=x|Y=y}=\sum_x x P_{X|Y}(x|y)
\\
\int_x x\Prob{X=x|Y=y}\text{d}x=\int_x x f_{X|Y}(x|y)\text{d}x
=\phi(Y)
\end{matrix}
$$

$$
\mathbb{E}[X|Y]=\phi(Y)
$$

$$
\mathbb{E}[X|Y_1,\dots,Y_N]=\sum_x x\Prob{X=x|Y_1=y_1,\dots,Y_N=y-N}=\phi(y_1,\dots,y_N)
$$

$$
\mathbb{E}[Y_1,\dots,Y_N]=\mathbb{E}[X|\mathcal{G}]=\phi(Y_1,\dots,Y_N)\text{ where }\mathcal{G}=Y_1,\dots,Y_N
$$

\subsection{Properties}

$X=c$ constantly $c\in\mathbb{R}\implies \mathbb{E}[X|\mathcal{G}]=c$

$X,Y$ integrable and $X\leq Y$ constantly $\implies\mathbb{E}[X|\mathcal{G}]\leq\mathbb{E}[Y|g]$ (monotonicity)

$\mathbb{E}[aX+bY|\mathcal{G}]=a\mathbb{E}[X|\mathcal{G}]+b\mathbb{E}[Y|\mathcal{G}]$

$|\mathbb{E}[X|\mathcal{G}]\leq\mathbb{E}[|x||g]$ constantly

\subsubsection{Law of total expectation}
$$
\mathbb{E}[X]=
\bigg\{
\begin{matrix}
\sum_y \mathbb{E}[X|y=y]\Prob{Y=y} \text{ Discrete case }
\\
\int_{\mathbb{R}}\mathbb{E}[X|y=y]f_Y(y)\text{d}y \text{ Continuous case }
\end{matrix}
$$

\section{Convergence of random variables}

$$
X_n:\Omega\mapsto \mathbb{R},
X:\Omega\mapsto \mathbb{R}
$$

\subsection{Almost Sure Convergence}

$$
\Prob{\{\omega\in\Omega:X_n(\omega)\rightarrow X(\omega)\}}=1
\iff
X_n \rightarrow^{\text{almost sure}}
$$

\subsection{$L^2$ Convergence}

$$
\mathbb{E}[|X-X_n|^2]\rightarrow 0
\iff X_n\rightarrow^{L^2}X
$$

\subsection{Convergence in Probability}

$$
\forall \epsilon>0, \prob{|X_n-X|>\epsilon}\rightarrow0\iff X_n\rightarrow^{\text{prob}}X
$$

\subsection{Weak Convergence(in distribution)}

$\forall f $ bounded continuous functions

$$
\mathbb{E}[f(X_n)]\rightarrow\mathbb{E}[f(X)]\iff X_n\rightarrow^{\text{w}}X
$$

all other three convergence $\implies$ weak convergence


\end{document}